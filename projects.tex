\documentclass[12pt]{article}

\usepackage[colorlinks]{hyperref}

\topmargin 0in
\textheight 8.5in
\textwidth  5.75in

\parskip 5pt

\newenvironment{routine}[2]
{\vspace{.25in}{\noindent\bf\hspace{0pt} #1}{\\ \noindent #2}
\begin{list}{}{
\renewcommand{\makelabel}[1]{{\bf ##1} \hfil} 
\itemsep 0pt plus 1pt minus 1pt
\leftmargin  1.2in
\rightmargin 0.0in
\labelwidth  1.1in
\itemindent  0.0in
\listparindent  0.0in
\labelsep    0.05in}
}{\end{list}}
%

\begin{document}

\title{Potential student projects}
\author{Mike~Giles}
%\date{}

\maketitle



\begin{abstract}
I think the overall OP2 project has the potential to provide a 
number of mini-projects which are suitable for undergraduate 
and MSc projects.
\end{abstract}


\section{Projects}


\subsection{FORTRAN port}

The existing implementation is for programs written in C/C++.  
The port to FORTRAN should be straightforward in principle, using 
the PGI FORTRAN CUDA compiler.  This might make a good 
third-year undergraduate project.

\subsection{OpenCL and SSE/AVX port}

The existing implementation is in CUDA for NVIDIA GPUs.  The port 
to OpenCL will hopefully be fairly straightforward, but the port 
to SSE/AVX vectors will probably require more experimentation and 
a good understanding of the Intel hardware and software; it's not 
clear whether this is best done using Ct or Intel's special SSE/AVX 
vector datatypes such as {\tt F32vec4}.


\subsection{Run-time optimisation of block sizes}

Each execution plan for parallel loops with indirection needs to 
decide on a block size, the number of set elements to be handled by
each thread block.

It is not very clear {\it a priori} what the right size should be.  
It mustn't be too large or else the shared memory requirements will 
exceed the amount available.  However, it is not clear that the best
choice is to maximise the size subject to this constraint.  By using 
a smaller size it will be possible for the run-time system to execute
more than one thread block on each multiprocessor, and this can have
significant benefits in overlapping the read/write performed at the 
beginning and end of each thread block, with the execution performed 
in the middle.

The idea for this project is to use run-time optimisation.  A number
of plans are constructed for each parallel loop, and the run-time execution
tries them all, keeps track of the execution time for each, and optimises
over time.  The optimisation will have to cope with the fact that there 
is probably a fair amoint of ``noise'' in the timing measurements.

I think this project fits in well with an important general theme in
scientific computing. Auto-tuning is becoming increasingly common 
(e.g.~FFTW, ATLAS) though usually I think this is done as a one-off 
optimisation (e.g.~when software is first installed) rather than
being done at run-time.


\subsection{Graph-based partitioning and renumbering}

In an OP application, the user specifies a number of sets and 
associated pointers between the sets.  These can be used to 
create a graph for each set by connecting elements which both
point to the same element of a different set, or which are pointed 
to by the same element of a different set.  

Graph-based partitioning
can then be used to do the high-level partitioning required for the
MPI distributed-memory parallelisation, and the low-level recursive
partitioning to renumber sets to maximise data reuse within the
thread blocks.

This is a more substantial project, and rather open-ended.


\subsection{Improved debugging tools}

The current single-threaded CPU implementation performs a number
of run-time checks on the user code, but there is a lot more that 
could be done to check the correctness of the code.

For example, the user declares the way in which each of the arguments
to the user's kernels is used, and also declares that the order in
which the set elements is processed does not affect the final outcome
(except for the limitations of finite precision arithmetic).  All of
this could be subjected to run-time checks.

\subsection{New applications}

New applications using the OP library would be welcome.



\end{document}
